%%%%%%%%%%%%%%%%%%
% WPC
\section{WPC}

\subsection{What is}
\textit{Work Pay Calculator} has the objective to simplify the \textbf{management} of the work done in terms of hour, cost and profits. In other hands WPC is an application that smartly stores hours done for a work and allows to automatize the process of hours pay calculation and \textit{occasional performance}\footnote{This is an italian fiscal document that certifies work for someone.} emission.

\subsection{What is not}
WPC is not a management tool for fiscal things, futhermore, for now, it's not an hour marker in terms of real-time start and stop counter (see \ref{subsec:enhancements}).

\subsection{Features}
\begin{enumerate}[label=(\alph*)]
\item Light-weight, simple and easy to use.
\item Easily syncronizable with cloud services: only the executable and the data storage is needed. Data storage needs to be syncronized only if the SQLite engine is chosed.
\item CLI inteface, in order to focus the objective to the application functionalities.
\item Highly configurable: personalize environment variables and achieve your needs.
\item Default answer configuration. Lot of configurable default answers, avoing boring data typing.
\end{enumerate}

\subsection{Examples}
\textbf{Scenario 1}: today I've worked for a customer 6 hours: 3 of coding, 1 learing a new technology and 2 to publish the work online. What the hell, I have to register this things before I forget them.

No problem, open WPC and insert a new work in a way like this:

\begin{lstlisting}
wpc> work: today 14:30 17:30 true
wpc> work: km? [0]: 
wpc> work: add? [0]: 
wpc> work: registry? [General work]: I've coded the
pinco pallo project
wpc> Work inserted successfully!
\end{lstlisting}

And you can repeate this procedure any time you want to increase the datail of hours done in this day.
